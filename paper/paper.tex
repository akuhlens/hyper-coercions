\documentclass[acmtog, authorversion, acmlarge]{acmart}
\usepackage{enumitem}
\usepackage{listings}
\usepackage{amsmath, amsthm}
\usepackage{stmaryrd}
\usepackage{proof} 
\usepackage{semantic}
\usepackage{xifthen}
\usepackage{subcaption}
\usepackage{wrapfig}

% Tell latex I really dislike it when you break my inline math!
\relpenalty=10000
\binoppenalty=10000

% I am a simple proof kinda guy
\newtheorem{thm}{Theorem}
\newtheorem{lem}{Lemma}
\newtheorem{cor}{Corollary}
% Metadata Information
%% \acmJournal{TOG}
%% \acmVolume{9}
%% \acmNumber{4}
%% \acmArticle{39}
%% \acmYear{2010}
%% \acmMonth{3}

% Copyright
%\setcopyright{acmcopyright}
%\setcopyright{acmlicensed}
%\setcopyright{rightsretained}
%\setcopyright{usgov}
%\setcopyright{usgovmixed}
%\setcopyright{cagov}
%\setcopyright{cagovmixed}

% DOI
%\acmDOI{0000001.0000001_2}

% Paper history
%\received{February 2007}
%\received{March 2009}
%\received[final version]{June 2009}
%\received[accepted]{July 2009}


% Document starts
\begin{document}
% Title portion
\title{Hyper-Coercions}
\author{Andre Kuhlenschmidt}
\affiliation{%
  \institution{Indiana University}
}

%% \author{Yafeng Wu}
%% \affiliation{%
%%   \institution{University of Virginia}
%%   \department{School of Engineering}
%%   \city{Charlottesville}
%%   \state{VA}
%%   \postcode{22903}
%%   \country{USA}
%% }

\renewcommand\shortauthors{A. Kuhlenschmidt}

\begin{abstract}
\end{abstract}

\maketitle

%% Types
%% the type meta variable

\newcommand{\T}{\ensuremath{T}}
\newcommand{\B}{\ensuremath{B}}
\newcommand{\I}{\ensuremath{I}}
\newcommand{\Fn}[3][]{\ensuremath{#2 #1 \to #1 #3}}
\newcommand{\Ref}[1]{\ensuremath{\texttt{Ref} \, #1}}
\newcommand{\Integer}{\ensuremath{\texttt{Int}}}
\newcommand{\Bool}{\ensuremath{\texttt{Bool}}}
\newcommand{\Dyn}{\ensuremath{\star}}


%% Blame labels
\newcommand{\Lbl}{\ensuremath{\ell}}

%% Hyper Coercion Syntax
\newcommand{\Prj}{\ensuremath{p}}
\newcommand{\pmt}{\ensuremath{\id}}
\newcommand{\Med}{\ensuremath{\mu}}
\newcommand{\Inj}{\ensuremath{i}}
\newcommand{\imt}{\ensuremath{\pmt}}
\newcommand{\HC}{\ensuremath{h}}
\newcommand{\hc}[5]{\ensuremath{\, #1 #2 #3 #4 #5 \,}}
\newcommand{\hcfail}[5]{\ensuremath{#1 #2 \failed{#3}{#4}{#5}}}

% Hyper-Coercion Syntax Shared with Coercions
\newcommand{\prj}[2][]{\ensuremath{#1 ?^{#2}}}
\newcommand{\inj}[1][]{\ensuremath{#1 !}}
\newcommand{\failed}[3]{\ensuremath{\bot^{#1 #2 #3}}}
\newcommand{\id}[1][]{\ensuremath{\iota_{#1}}}
\newcommand{\fn}[3][]{\ensuremath{#2 #1 \to #1 #3}}
\newcommand{\rf}[2]{\ensuremath{\texttt{Ref} \, #1 \, #2}} 

% Coercions Syntax
\newcommand{\Crcn}{\ensuremath{c}}
\newcommand{\SE}{\ensuremath{s}}
\newcommand{\Int}{\ensuremath{m}}
\newcommand{\Fnl}{\ensuremath{f}}
\newcommand{\Seq}[2]{\ensuremath{#1 ; #2}}

%% Consistency, Shallow Consistency, and Shallow Inconsistency
\newcommand{\sct}[1][\;]{\ensuremath{#1 \overline{\#} #1}}
\newcommand{\sic}[1][\;]{\ensuremath{#1 \# #1}}

%% Hyper Coercion and Coercion Well Typed Judgment
\newcommand{\cwt}[3]{\ensuremath{\vdash \, #1 \, : \, #2 \Rightarrow #3}}

%% Cast and Make Coercion Relation
\newcommand{\cast}[3][\Lbl]{\ensuremath{#2 \Rightarrow^{#1} #3}}
\newcommand{\mkcrcn}[3][\Lbl]{\ensuremath{\langle \cast[#1]{#2}{#3} \rangle}}
\newcommand{\mkIcrcn}[3][\Lbl]{\ensuremath{\{ \cast[#1]{#2}{#3} \}}}
\newcommand{\mkmed}[3][\Lbl]{\ensuremath{\langle \widehat{\cast[#1]{#2}{#3}} \rangle}}

%% Compose and Compose Mediating Coercions
\newcommand{\cmp}[2]{\ensuremath{#1 \fatsemi #2}}
\newcommand{\cmpm}[2]{\cmp{#1}{#2}}

%% Metafunctions
\newcommand{\depth}[1]{\ensuremath{\mid #1 \mid}}
\newcommand{\Max}[1]{\ensuremath{\texttt{max}(#1)}}

%% Compiling to / from
\newcommand{\interp}[3]{%
  \ensuremath{\llbracket #1 \rrbracket_{#3}^{#2}}}

\section{Syntax, Typing, and Composition}
\label{sec:hc}

Hyper-Coercions are a densely packed version of space-efficient
coercions in the style of \citet{Siek:2015ab} which we hope to use to
increase the performance of the coercions implementation of gradual
typing in the Schml compiler. The main intuition behind their
formulation is that the longest sequence of coercions in
space-efficient normal form is a series of a projection, then some
cast mediating between two possibly equal types, followed a injection;
the hyper-coercion structure can represent this longest chain of
coercions in a single hyper-coercion node. A hyper-coercion (\HC \; in
Figure~\ref{fig:hcSyntax}) is a 5-tuple that contain a projection
(\Prj), the source type and target types (\T{}s) of the intermediating
coercion (\Med), the intermediating coercion itself, and finally an
injection (\Inj). Figure~\ref{fig:hcTyping} shows the typing judgments
for hyper-coercions and their subcomponent coercions.  We can seen
that a hyper-coercion, \hc{\Prj}{\T_2}{\Med}{\T_3}{\Inj}, is
well-typed cast from $\T_1$ to $\T_2$ if there is well-typed
projection (\Prj) from $\T_1$ to $\T_2$, a well-typed intermediating
cast (\Med) from $\T_2$ to $\T_3$, and well typed injection from
$\T_2$ to $\T_3$.

If a previous composition operation (shown in
Figure~\Ref{fig:hcComposition}) resulted in a failed coercion then the
alternative \emph{failed hyper-coercion}
(\hcfail{\Prj}{\I}{\I_{2}}{\Lbl}{\I_{3}}) is used to signify that
after a projection (\Prj) to type (\T), the failed hyper-coercion will
report an error due to inconsistency between injectable (non-dynamic)
types $\I_1$ and $\I_2$. This failure will blame \Lbl, but the
projection must be preserved because it also has the potential before
this error is reported. A failed coercion is well-typed from $\T_1$ to
$\T_2$ if the projection is well-typed from $\T_1$ to $\I_1$, $\I_1$
is shallow consistent with $\I_2$, and $I_2$ is shallow inconsistent
with $\I_2$.

The rest of typing judgments are standard. Syntactically, projections
and injections differ in that they no longer contain the types that they
project to and inject from. Instead their source and target types
are shared with the overall hyper-coercion. This is merely to reduce
the complexity of composition rules, and reflects the mentality of
considering a sequence of casts as a single entity.

Make coercion takes any two types and a blame label and returns a
hyper-coercion.
\begin{figure}[tbh]
  \small
  \begin{subfigure}{.5\textwidth}
    \[
    \begin{array}{llcl}
      \textbf{Base Types} &
      \B & := & \Integer \mid \Bool \\
      \textbf{Types} &
      \T    & ::= & \B \mid \Fn{\T}{\T} \mid \Ref{\T} \mid \Dyn \\
      \textbf{Injectable Types} &
      \I    & ::= & \B \mid \Fn{\T}{\T} \mid \Ref{\T} \\
      \textbf{Projections}&
      \Prj  & ::= & \pmt \mid \prj{\Lbl} \\
      \textbf{Injections}&
      \Inj  & ::= & \imt \mid \; \inj \\
      \textbf{Intermediates}&
      \Med  & ::= & \id \mid \fn{\HC}{\HC} \mid \rf{\HC}{\HC}\\
      \textbf{Hyper Coercions}&
      \HC   & ::= & \hc{\Prj}{\T}{\Med}{\T}{\Inj} \mid \hcfail{\Prj}{\T}{\I}{\Lbl}{\I}
    \end{array}
    \]
    \caption{Syntax of Hyper Coercions}
    \label{fig:hcSyntax}
  \end{subfigure}%
  \begin{subfigure}{0.5\textwidth}
    \begin{gather*}
      \inference{}{\B \sct \B}
      \quad
      \inference{}{\Fn{\T_{11}}{\T_{12}} \sct \Fn{\T_{21}}{\T_{22}}}
      \\[2ex]
      \inference{}{\Ref{\T_{1}} \sct \Ref{\T_{2}}}
      \quad
      \inference{}{\Dyn \sct \T}
      \quad
      \inference{}{\T \sct \Dyn}
      \\[2ex]
      \inference{\neg (\T_1 \sct \T_2)}{\T_1 \sic \T_2}
    \end{gather*}
    \caption{Shallow Consistency(\sct) and Inconsistency (\sic)}
    \label{fig:shallowConsistency}
  \end{subfigure}
  \begin{subfigure}{.5\textwidth}
    \begin{gather*}
      %% \inference{}{\cwt{\pmt}{\T}{\T}}
      %% \\[2ex]
      \inference{}{\cwt{\prj{\Lbl}}{\Dyn}{\I}}
      \quad
      \inference{\cwt{\HC_1}{\T_{3}}{\T_{1}} &
                 \cwt{\HC_2}{\T_{2}}{\T_{4}}}
                {\cwt{\fn{\HC_1}{\HC_2}}
                     {\Fn{\T_{1}}{\T_{2}}}
                     {\Fn{\T_{3}}{\T_{4}}}}
      \\[2ex]
      \inference{}{\cwt{\inj}{\I}{\Dyn}}      
      \quad
      \inference{\cwt{\HC_1}{\T_{2}}{\T_{1}} &
                 \cwt{\HC_2}{\T_{1}}{\T_{2}}}
                {\cwt{\rf{\HC_1}{\HC_2}}
                     {\Ref{\T_{1}}}
                     {\Ref{\T_{2}}}}
      \\[2ex]
      \inference{}{\cwt{\id}{\T}{\T}}
      \quad          
      \inference{\cwt{\Prj}{\T_1}{\I_1} &
                 \I_1 \sct \I_2 &
                 \I_2 \sic \I_3}
                {\cwt{\hcfail{\Prj}{\I_1}{\I_2}{\Lbl}{\I_3}}{\T_1}{\T_2}}
      \\[2ex]
      \inference{\cwt{\Prj}{\T_1}{\T_2} &
                 \cwt{\Med}{\T_2}{\T_3} &
                 \cwt{\Inj}{\T_3}{\T_4}}
                {\cwt{\hc{\Prj}{\T_2}{\Med}{\T_3}{\Inj}}{\T_1}{\T_4}}
    \end{gather*}
    \caption{Typing Judgment}
    \label{fig:hcTyping}
  \end{subfigure}%
  \begin{subfigure}{.5\textwidth}
    \begin{align*}
      \mkcrcn{\T}{\T} &= \hc{\pmt}{\T}{\id}{\T}{\imt} \\
      %
      \mkcrcn{\Dyn}{\I} &= \; \hc{\prj{\Lbl}}{\I}{\id}{\I}{\imt} \\
      %
      \mkcrcn{\I}{\Dyn} &= \hc{\pmt}{\I}{\id}{\I}{\inj} \\
      %
      \mkcrcn{\Fn[\!]{\T_{1}}{\T_{2}}}{\Fn[\!]{\T_{3}}{\T_{4}}} &=
         \hc{\pmt}{\Fn[\!]{\T_{1}}{\T_{2}}}
            {(\fn[\!]{\mkcrcn{\T_{3}}{\T_{1}}}{\mkcrcn{\T_{2}}{\T_{4}}})} 
            {\Fn[\!]{\T_{3}}{\T_{4}}}{\imt} \\ 
            & \hspace{5.3em} \texttt{if} \; % 
            \Fn[\!]{\T_{1}}{\T_{2}} \neq \Fn[\!]{\T_{3}}{\T_{4}} \\
      %
      \mkcrcn{\Ref{\T_1}}{\Ref{\T_2}} &=
         \hc{\pmt}{\Ref{\T_{1}}}
            {(\rf{\mkcrcn{\T_{2}}{\T_{1}}}{\mkcrcn{\T_{1}}{\T_{2}}})} 
            {\Ref{\T_{2}}}{\imt} \\
            & \hspace{5.3em} \texttt{if} \; % 
            \Ref{\T_1} \neq \Ref{\T_2} \\
      %
      \mkcrcn{\I_1}{\I_2} &= \hcfail{\pmt}{\I_1}{\I_1}{\Lbl}{\I_2} %
      \hspace{.6em} \texttt{if} \; \I_1 \sic \I_2
    \end{align*}
    \caption{Make Coercion}
    \label{fig:hcMakeCoercion}
  \end{subfigure}
  \begin{subfigure}{.5\textwidth}
    \begin{align*}
      %
      \cmp{\hc{\pmt}{\Dyn}{\id}{\Dyn}{\imt}}{\HC} &= \HC \\
      %
      \cmp{\hc{\Prj}{\I_1}{\Med}{\I_2}{\inj}} %
          {\hc{\pmt}{\Dyn}{\id}{\Dyn}{\imt}} %
      &=  \hc{\Prj}{\I_1}{\Med}{\I_2}{\inj} \\
      % Comp_hc_no_inj_prj
      \cmp{\hc{\Prj}{\I_1}{\Med_1}{\I_2}{\imt}} %
          {\hc{\pmt}{\I_2}{\Med_2}{\I_3}{\Inj}} %
      &=  \hc{\Prj}{\I_1}{\cmpm{\Med_1}{\Med_2}}{\I_3}{\Inj} \\
      % Comp_hc_inj_prj_ok
      \cmp{\hc{\Prj}{\I_1}{\Med_1}{\I_2}{\inj}} %
          {\hc{\prj{\Lbl}}{\I_3}{\Med_2}{\I_4}{\Inj}} %
      &=  \hc{\Prj}{\I_1}{\cmpm{\cmpm{\Med_1}{\Med_3}}{\Med_2}}{\I_4}{\Inj} %
          & \texttt{if} \;
          \mkcrcn{\I_2}{\I_3} = \hc{\pmt}{\I_2}{\Med_3}{\I_3}{\imt}   \\
      % Comp_hc_inj_prj_fail
          \cmp{\hc{\Prj}{\I_1}{\Med_1}{\I_2}{\inj}} %
          {\hc{\prj{\Lbl}}{\I_3}{\Med_2}{\I_4}{\Inj}} %
      &= \hcfail{\Prj}{\I_1}{\I_2}{\Lbl}{\I_3} %
          & \texttt{if} \;
          \mkcrcn{\I_2}{\I_3} = \hcfail{\pmt}{\I_2}{\I_2}{\Lbl}{\I_3}   \\
      % Comp_fail_l
      \cmp{\hcfail{\Prj}{\I_1}{\I_2}{\Lbl}{\I_3}} %
          {\HC} %
      &=  \hcfail{\Prj}{\I_1}{\I_2}{\Lbl}{\I_3} \\
      % Comp_fail_r_no_inj_prj
      \cmp{\hc{\Prj}{\I_1}{\Med}{\I_2}{\imt}} %
          {\hcfail{\pmt}{\I_2}{\I_3}{\Lbl}{\I_4}} %
      &=  \hcfail{\Prj}{\I_1}{\I_3}{\Lbl}{\I_4} \\
      % Comp_fail_r_inj_prj_ok
      \cmp{\hc{\Prj}{\I_1}{\Med}{\I_2}{\inj}} %
          {\hcfail{\prj{\Lbl_1}}{\I_3}{\I_4}{\Lbl_2}{\I_5}} % 
      &=  \hcfail{\Prj}{\I_1}{\I_4}{\Lbl_2}{\I_5} %
          & \texttt{if} \;
          \mkcrcn[\Lbl_1]{\I_2}{\I_3} = \hc{\pmt}{\I_2}{\Med_3}{\I_3}{\imt} \\
      % Comp_fail_r_inj_prj_fail
      \cmp{\hc{\Prj}{\I_1}{\Med}{\I_2}{\inj}} %
          {\hcfail{\prj{\Lbl_1}}{\I_3}{\I_4}{\Lbl_2}{\I_5}} %
      &=  \hcfail{\Prj}{\I_1}{\I_2}{\Lbl_1}{\I_3} %
          & \texttt{if} \;
          \mkcrcn[\Lbl_1]{\I_2}{\I_3} = \hcfail{\pmt}{\I_2}{\I_2}{\Lbl_1}{\I_3}
    \end{align*}
    \caption{Compose Hyper-Coercions}
    \label{fig:composeHC}
  \end{subfigure}%
  \begin{subfigure}{.5\textwidth}
    \begin{align*}
      \cmpm{\id}{m} &= m\\
      \cmpm{m}{\id} &= m & m \neq \id \\
      \cmpm{\fn{\HC_1}{\HC_2}}{\fn{\HC_3}{\HC_4}} %
      &= \fn{\cmp{\HC_3}{\HC_1}}{\cmp{\HC_2}{\HC_4}}\\
      \cmpm{\rf{\HC_1}{\HC_2}}{\rf{\HC_3}{\HC_4}} %
      &= \rf{\cmp{\HC_3}{\HC_1}}{\cmp{\HC_2}{\HC_4}}\\ 
    \end{align*}
    \caption{Compose Intermediates}
    \label{fig:composeMed}
  \end{subfigure}
  \begin{subfigure}{.5\textwidth}
    \begin{align*}
      \depth{\B} &= 0 \\
      \depth{\Dyn} &= 0 \\
      \depth{\Fn{\T_1}{\T_2}} &= 1 + \Max{\depth{\T_1}, \depth{\T_2}}\\
      \depth{\Ref{\T}} &= 1 + \depth{\T}
    \end{align*}
    \caption{Type Depth}
    \label{fig:hctydepth}
  \end{subfigure}%
  \begin{subfigure}{.5\textwidth}
    \begin{align*}
      \depth{\hc{\Prj}{\T_1}{\Med}{\T_2}{\Inj}} %
      &= \Max{\depth{\T_1},\depth{\Med},\depth{\T_2}}\\
      \depth{\hcfail{\Prj}{\T}{\I_1}{\Lbl}{\I_2}} %
      &= \; \depth{\T_1} \\
      \depth{\id} &= 0\\
      \depth{\fn{\HC_1}{\HC_2}} &= 1 + \Max{\depth{\HC_1}, \depth{\HC_2}}\\
      \depth{\rf{\HC_1}{\HC_2}} &= 1 + \Max{\depth{\HC_1}, \depth{\HC_2}}
    \end{align*}
    \caption{Hyper-Coercion Depth}
    \label{fig:hctydepth}
    \end{subfigure}
  \caption{The Hyper-Coercion Representation}
  \label{fig:hc}
\end{figure}

\clearpage

\section{Composition of Hyper-Coercions Preserves Well-Typing}

In order to prove that composition of hyper-coercion preserve
the well-typing of hyper-coercions, we must first show a
few properties about the make hyper-coercion relation. 

\begin{lem}
  \label{lem:mk_hc_wt}
  (Make hyper-coercion produces well-typed hyper-coercions.)\\
  Forall $\T_1$ $\T_2$ \Lbl \, \HC, \,
  $\mkcrcn{\T_1}{\T_2} = \HC$ implies that \cwt{\HC}{\T_1}{\T_2}
\end{lem}
\begin{proof}
  Proof by induction on the depth of \HC, and inversion
  on the construction of \HC. 
\end{proof}

\begin{lem}
  \label{lem:mk_hc_sym}
  (make coercion is defined over symmetric arguments.)\\
  Forall $\T_1$ $\T_2$ $\Lbl$ $\HC_1$, \,
  $\mkcrcn{\T_1}{\T_2} = \HC_1$ \, implies that
  there exists an $\HC_2$ such that
  $\mkcrcn{\T_2}{\T_1} = \HC_2$ and
  $\depth{\HC_2} \le max \depth{\T_2} \depth{\T_1}$.
\end{lem}
\begin{proof}
  Proof by induction on the depth of coercions, and inversion
  on the construction of the coercion. 
\end{proof}

\begin{lem}
  \label{lem:mk_hc_fn}
  (Make hyper-coercion is deterministic.)\\
  Forall $\T_1$ $\T_2$ $\Lbl$ $\HC_1$ $\HC_2$, \,
  $\mkcrcn{\T_1}{\T_2} = \HC_1$ and
  $\mkcrcn{\T_1}{\T_2} = \HC_2$ implies that
  $\HC_1 = \HC_2$.
\end{lem}
\begin{proof}
  Proof by induction on the depth of the types and
  inversion on the constructions of coercions. 
\end{proof}

\begin{lem}
  \label{lem:mk_hc_fn}
  (Make hyper-coercion is a total function.)\\
  Forall $\T_1$ $\T_2$ $\Lbl$,
  there exists an $\HC$ such that
  $\mkcrcn{\T_1}{\T_2} = \HC$ and
  $\depth{\HC} \le \depth{\T_1} \depth{\T_2}$.
\end{lem}
\begin{proof}
  Proof by induction on $\T_1$ and case analysis of $\T_2$
  using Lemma~\ref{lem:mk_hc_sym} where type arguments
  are switched in function and reference coercion construction. 
\end{proof}

\begin{lem}
  \label{lem:mk_hc_dec}
  (Make failed hyper-coercion is decidable.)\\
  Forall $\I_1$ $\I_2$ $\Lbl$,
  either $\I_1 \sct \I_2$ and there exists and $\Med$
  such that $\mkcrcn{\I_1}{\I_2} = \hc{\pmt}{\I_1}{\Med}{\I_2}{\imt}$
  and $\depth{\hc{\pmt}{\I_1}{\Med}{\I_2}{\imt}}
  \le max \depth{\I_1} \depth{\I_2}$,
  or $\I_1 \sic \I_2$ and
  $\mkcrcn{\I_1}{\I_2} = \hcfail{\pmt}{\I_1}{\I_1}{\Lbl}{\I_2}$.
\end{lem}
\begin{proof}
  Proof by case analysis on the decidability of shallow consistency,
  use of Lemma~\ref{lem:mk_hc_fn}, and inversion on the construction
  of the coercion. 
\end{proof}

Because we are choosing to view the maximal sequence of
space-efficient coercions as standard size of a single hyper-coercion
proofs about hyper-coercions tend to have very deep case analysis and
reasoning before we are able to use the inductive hypothesis. This can
lead to repetitive proving of many of the same properties. To
alleviate this we break our Theorem~\ref{thm:composeWDF} into two
Lemmas~\ref{lem:thm:help} and~\ref{lem:composeIWDF} which end up being straightforward and relatively short.

Intuitively, Lemma~\ref{lem:thm:help} states that the proof obligation
of the inductive step of a hypothetical inductive proof of
Theorem~\ref{thm:composeWDF} can be proven given the inductive
hypothesis of the proof of Lemma~\ref{lem:composeIWDF}. Since
compose hyper-coercions and compose intermediates are mutually
recursive a proof that either one of them is well behaved is
enough to ``tie the knot'', but would normally have to induct
all the way through the other definition in order to do so.

\begin{lem}
  \label{lem:thm:help}
  (Theorem~\ref{thm:composeWDF} inductive step is implied by Lemma~\ref{lem:composeIWDF} inductive hypothesis.)
  Forall $n$, $\HC_1$ and $\HC_2$ such that\;
  $\depth{\HC_1} < n$;
  \, $\depth{\HC_2} < n$; \,
  $\cwt{\HC_1}{\T_1}{\T_2}$;
  and $\cwt{\HC_2}{\T_2}{\T_3}$ imply that the assumption:
  ``Forall $\Med_1$ and $\Med_2$, such that
  $\depth{\Med_1} \, \le n$, \; $\depth{\Med_2} \, \le n$, \;
  $\cwt{\Med_1}{\T_1}{\T_2}$, and $\cwt{\Med_2}{\T_2}{\T_3}$,
  imply that there exists an $\Med_3$ such that
  $\cwt{\Med_3}{\T_1}{\T_3}$, \,
  $\cmp{\Med_1}{\Med_2} = \Med_3$, \,
  forall $\Med_{3}^{\prime}$ such that
  $\cmp{\Med_1}{\Med_2} = \Med_{3}^{\prime}$
  implies that $\Med_{3}^{\prime} = \Med_{3}$,
  and $\depth{\Med_3} \le \depth{\Med_1} \depth{\Med_2}$''
  implies that there exists an $\HC_3$ such that
  $\depth{\HC_3} \, \le max \depth{\HC_1} \, \depth{\HC_2}$, \,
  $\cwt{\HC_3}{\T_1}{\T_3}$, \, $\cmp{\HC_1}{\HC_2} = \HC_3$, and
  forall $H_{3}^{\prime}$, \; $\cmp{\HC_1}{\HC_2} = H_{3}^{\prime}$
  implies that $H_{3}^{\prime} = H_{3}$.
\end{lem}
\begin{proof}
  We proceed by case analysis on the well-typed hyper-coercions
  $\HC_1$ and $\HC_2$. Whenever an non-identity project and
  inject sequence meet we case on whether the two types are
  shallow-consistent or not. This allows us to decide whether
  the make hyper-coercion call will result in a hyper-coercion
  or a failed hyper-coercion in accordance with
  Lemma~\ref{lem:mk_hc_dec}.

  As an example let's take the hardest case where two non-failed
  hyper-coercions are compose and a call to make coercion is necessary.
  (i.e. $\HC_1 = \hc{\Prj}{\I_{1}}{\Med_1}{\I_{2}}{\inj}$
  and $\HC_2 = \hc{\prj{\Lbl}}{\I_{3}} {\Med_1}{\I_{4}}{\Inj}$)
  According to Lemma~\ref{lem:mk_hc_dec} it will either be
  the case that
  $\I_{2} \sct \I_{3}$ and
  there exists an $\Med_3$ such that
  $\mkcrcn{\I_{2}}{\I_{3}} = \hc{\pmt}{\I_{2}}{\Med_3}{\I_{2}}{\imt}$
  and $\depth{\Med_3} \le max \depth{\I_{2}}\depth{\I_{3}}$
  or
  $\I_{2} \sic \I_{3}$ and
  $\mkcrcn{\I_{2}}{\I_{3}} = \hcfail{\pmt}{\I_{2}}{\I_2}{\Lbl}{\I_{3}}$
  \begin{itemize}
  \item When $\I_{2} \sct \I_{3}$,
      \begin{itemize}
      \item We use our assumption about the composition of
        intermediating coercions to show there exists an $\Med_4$ from
        composing $\Med_1$ and $\Med_3$.  In the case of $\Med_1$,
        we derive \cwt{\Med_1}{I_1}{I_2} and $\depth{\Med_1} \le n$
        from the bounds placed on the hyper-coercion that contained it.
        In the case of $\Med_3$ we derive \cwt{\Med_3}{\I_2}{\I_2} from
        Lemma~\ref{lem:mk_hc_wt} and $\depth{\Med_3} \le n$ by
        reasoning about the bounds set by:
        $\depth{\Med_3} \le max \depth{\I_{2}}\depth{\I_{3}}$, \, 
        $\I_2 \in \HC_1$, \,
        $\I_3 \in \HC_2$, \,
        $\depth{\HC_1} < 1 + n$,\,
        and $\depth{\HC_2} < 1 + n$. 
      \item We again use our assumption about the composition
        of intermediates to prove the existence of $\Med_5$ which
        results from the composition of $\Med_4$ and $\Med_2$.
        The use of $\Med_4$ with this assumption is justified
        because when $\Med_4$ was proven to exists we also
        proved \cwt{\Med_4}{\I_1}{\I_3} and
        $\depth{\Med_4} \le max \depth{\Med_1} \depth{\Med_3}$ where
        $\Med_1$ and $\Med_3$ have both already been shown to be
        less than $n$. The use of $\Med_2$ with this assumption
        is justified for the same reason that $\Med_1$ was.
      \item We can now chose
        $\HC_3 = \hc{\Prj}{\I_1}{\Med_5}{\I_2}{\Inj}$ and the
        rest of the proof obligations are trivial. 
      \end{itemize}
    \item When $\I_{2} \sic \I_{3}$, choose
      $\HC_3 = \hcfail{\Prj}{\I_{1}}{\I_2}{\Lbl}{\I_{3}}$ and the
      remaining proof obligations are trivial.
  \end{itemize}
\end{proof}

Now that we have proven this it is straight forward to show
that compose intermediating coercions is a type preserving
deterministic function. 

\begin{lem}
  \label{lem:composeIWDF}
  (Compose intermediates is a type-preserving deterministic total function.)
  Forall $n$ $\Med_1$ and $\Med_2$, such that
  $\depth{\Med_1} \, \le n$, \; $\depth{\Med_2} \, \le n$, \;
  $\cwt{\Med_1}{\T_1}{\T_2}$, and $\cwt{\Med_2}{\T_2}{\T_3}$,
  there exists an $\Med_3$ such that
  $\cwt{\Med_3}{\T_1}{\T_3}$, \,
  $\cmp{\Med_1}{\Med_2} = \Med_3$, \,
  forall $\Med_{3}^{\prime}$, \; $\cmp{\Med_1}{\Med_2} = \Med_{3}^{\prime}$
  implies that $\Med_{3}^{\prime} = \Med_{3}$,
  and $\depth{\Med_3} \le \depth{\Med_1} \depth{\Med_2}$.
\end{lem}
\begin{proof}
  Proof by induction on $n$. 
  \begin{itemize}
  \item Proving the base case is trivial.
  \item For the inductive step we proceed by case
    analysis of the well-typed $\Med_1$
    and $\Med_2$, and use the inductive hypothesis to
    justify the use of Lemma~\ref{lem:thm:help}. 

    As an example consider the case where
    $\Med_1 = \fn{\HC_1}{\HC_2}$ and
    $\Med_2 = \fn{\HC_3}{\HC_4}$.
    %  
    Note that it is easy to show that $\HC_{1-4}$ are well
    typed by inversion of the well-typing on there containing
    intermediating coercions.
    %
    Also note that from simplification and some reasoning
    about the behavior of $max$ it is easy to show that
    $\depth{\HC_{1-4}} < 1 + n$.
    %
    As such each are justified for use Lemma~\ref{lem:thm:help}
    aided by the inductive hypothesis.
    \begin{itemize}
    \item We Lemma~\ref{lem:thm:help} once with $\HC_3$ and $\HC_1$
      and prove that there is an $HC_5$.
    \item We Lemma~\ref{lem:thm:help} again with $\HC_2$ and $\HC_4$
      and prove that there is an $HC_6$.
    \end{itemize}
    We choose $\Med_3 = \Fn{\HC_5}{\HC_6}$ and the rest of the
    obligations are trivial.
  \end{itemize} 
\end{proof}

Theorem~\ref{thm:composeWDF} follows immediately from
Lemma~\ref{lem:thm:help} and Lemma~\ref{lem:composeIWDF}.

\begin{thm}
  \label{thm:composeWDF}
  (Compose is a type-preserving deterministic total function.)  Forall
  $n$ $\HC_1$ and $\HC_2$ such that $\depth{\HC_1} < n$, \,
  $\depth{\HC_2} < n$, \, $\cwt{\HC_1}{\T_1}{\T_2}$, and
  $\cwt{\HC_2}{\T_2}{\T_3}$, and , there exists an $\HC_3$ such that
  $\depth{\HC_3} \, \le max \depth{\HC_1} \; \depth{\HC_2}$, \,
  $\cwt{\HC_3}{\T_1}{\T_3}$, \, $\cmp{\HC_1}{\HC_2} = \HC_3$, and
  forall $H_{3}^{\prime}$, \; $\cmp{\HC_1}{\HC_2} = H_{3}^{\prime}$
  implies that $H_{3}^{\prime} = H_{3}$.
\end{thm}
\begin{proof}
  This is implied by the instantiation of Lemma~\ref{lem:thm:help}
  using Lemma~\ref{lem:composeIWDF}.
\end{proof}

\begin{cor}
  (Compose hyper-coercions preserves well-typing.)\\
  Forall, $\HC_1$ $\HC_2$ $\HC_3$ $\T_1$ $\T_2$ $\T_3$, \;
  \cwt{\HC_1}{\T_1}{\T_2}, \; \cwt{\HC_2}{\T_2}{\T_3}, and
  \cmp{\HC_1}{\HC_2}{\HC_3} \; imply \cwt{\HC_3}{\T_1}{\T_3}.
\end{cor}
\begin{proof}
  Instantiation of Theorem~\ref{thm:composeWDF}.
\end{proof}

\clearpage

\section{The Hyper-Coercions / Coercion Isomorphism}
\begin{figure}[tbh]
  \centering
  \begin{subfigure}{.5\textwidth}
    \[
    \begin{array}{llcl}
      \textbf {Coercions}&
      \Crcn & ::= & \id[\T] \mid \prj[\I]{\Lbl} \mid \inj[\I] \mid %
                    \Seq{\Crcn}{\Crcn} \mid \\
      &     &     & \failed{\I}{\Lbl}{\I} \mid %
                    \fn{\Crcn}{\Crcn} \mid \rf{\Crcn}{\Crcn}\\
      \textbf{Space-Efficient Coercions}&
      \SE  & ::= & \id[\Dyn] \mid \Seq{\prj[\I]{\Lbl}}{\Fnl} \mid \Fnl \\
      \textbf{Final Coercions}&
      \Fnl  & ::= & \failed{\I}{\Lbl}{\I} \mid \Seq{\Int}{\inj[\I]} \mid \Int \\
      \textbf{Intermediate Coercions}&
      \Int  & ::= & \id[\I] \mid \fn{\SE}{\SE} \mid \rf{\SE}{\SE}\\
    \end{array}
    \]
    \caption{Syntax of Space Efficient Lazy-D Coercions}
    \label{fig:hcSyntax}
  \end{subfigure}
  \begin{subfigure}{.5\textwidth}
    \begin{align*}
      \cmp{\Seq{\Int}{\inj[\I_1]}}
          {\Seq{\prj[\I_2]{\Lbl}}{\Fnl}} &=%
          \cmp{\Int}
              {\cmp{\mkcrcn{\I_1}{\I_2}}
                   {\Fnl}}%
    \\
      \cmp{\fn{\SE_1}{\SE_2}}
          {\fn{\SE_3}{\SE_4}} &=
          \fn{\cmp{\SE_3}{\SE_1}}
             {\cmp{\SE_2}{\SE_4}}
    \\
    \cmp{\rf{\SE_1}{\SE_2}}{\rf{\SE_3}{\SE_4}} &=
    \rf{(\cmp{\SE_3}{\SE_1})}{(\cmp{\SE_2}{\SE_4})}
    \\
    \cmp{(\Seq{\prj[\I]{\Lbl}}{\Fnl})}{\SE} &=
    \Seq{\prj[\I]{\Lbl}}{(\cmp{\Fnl}{\SE})}
    \\
    \cmp{\Int_1}{(\Seq{\Int_2}{\inj[\I]})} &=
    \Seq{(\cmp{\Int_1}{\Int_2})}{\inj[\I]}
    \\
    \cmp{\id[\T]}{\SE} &= \SE
    \\
    \cmp{\SE}{\id[\T]} &= \SE
    \\
    \cmp{\Int}{\failed{\I_1}{\Lbl}{\I_2}} &=
    \failed{\I_1}{\Lbl}{\I_2}
    \\
    \cmp{\failed{\I_1}{\Lbl}{\I_2}}{\SE} &=
    \failed{\I_1}{\Lbl}{\I_2}
    \end{align*}
    \caption{Compose Space-Efficient Coercions}
    \label{fig:composeMed}
  \end{subfigure}%
  \begin{subfigure}{.5\textwidth}
    \begin{gather*}
      \inference{}{\cwt{\id[\T]}{\T}{\T}}
      \quad
      \inference{\cwt{\Crcn_1}{\T_1}{\T_2} & \cwt{\Crcn_2}{\T_2}{\T_3}}
                {\cwt{\Seq{\Crcn_1}{\Crcn_2}}{\T_1}{\T_3}}
      \\[2ex]
      \inference{}{\cwt{\prj[\I]{\Lbl}}{\Dyn}{\I}}
      \quad
      \inference{\cwt{\Crcn_1}{\T_{3}}{\T_{1}} & \cwt{\Crcn_2}{\T_{2}}{\T_{4}}}
                {\cwt{\fn{\Crcn_1}{\Crcn_2}}
                     {\Fn{\T_{1}}{\T_{2}}}
                     {\Fn{\T_{3}}{\T_{4}}}}
      \\[2ex]
      \inference{}{\cwt{\inj}{\I}{\Dyn}}
      \quad
      \inference{\cwt{\Crcn_1}{\T_{2}}{\T_{1}} &
                 \cwt{\Crcn_2}{\T_{1}}{\T_{2}}}
                {\cwt{\rf{\Crcn_1}{\Crcn_2}}
                     {\Ref{\T_{1}}}
                     {\Ref{\T_{2}}}}
      \\[2ex]
      \inference{\I_3 \sim \I_1 & \I_2 \sim \T}
                {\cwt{\failed{\I_1}{\Lbl}{\I_2}}{\I_3}{\T}}
    \end{gather*}
    \caption{Coercions (\Crcn \, and \SE) Typing Judgment}
    \label{fig:typingJudgment}
  \end{subfigure}
  \begin{subfigure}{.5\textwidth}
    \begin{align*}
      \interp{\hc{\prj{\Lbl}}{\T_2}{\Med}{\T_3}{\inj}}
             {\T_1}
             {\T_4}
      &=
      \Seq{\prj[\T_2]{\Lbl}}
          {\Seq{\interp{\Med}{\T_2}{\T_3}}{\inj[\T_3]}}
      \\
      \interp{\hc{\pmt}{\T_1}{\Med}{\T_2}{\inj}}
             {\T_1}{\T_3}
      &=
      \Seq{\interp{\Med}{\T_1}{\T_2}}{\inj[\T_2]}
      \\
      \interp{\hc{\prj{\Lbl}}{\T_2}{\Med}{\T_3}{\imt}}
             {\T_1}{\T_3}
      &=
      \Seq{\prj[\T_1]{\Lbl}}{\interp{\Med}{\T_2}{\T_3}}
      \\
      \interp{\hcfail{\prj{\Lbl_1}}{\T_2}{\T_3}{\Lbl_2}{\T_4}}{\T_1}{\T_5}
      &=
      \Seq{\prj[\T_1]{\Lbl}}{\failed{\T_2}{\Lbl_2}{\T_3}}
      \\
      \interp{\hcfail{\pmt}{\T_1}{\T_2}{\Lbl_2}{\T_3}}
             {\T_1}{\T_4}
             &=
      \failed{\T_2}{\Lbl_2}{\T_3}
      \\
      \interp{\id}{\T}{\T} &= \id[\T]
      \\
      \interp{\fn{\HC_1}{\HC_2}}
             {\Fn{\T_1}{\T_2}}{\Fn{\T_3}{\T_4}}
             &=
      \fn{\interp{\HC_1}{\T_3}{\T_1}}
         {\interp{\HC_2}{\T_2}{\T_4}}
      \\
      \interp{\rf{\HC_1}{\HC_2}}
             {\Ref{\T_1}}{\Ref{\T_2}}
             &=
      \rf{\interp{\HC_1}{\T_2}{\T_1}}
         {\interp{\HC_2}{\T_1}{\T_2}}
    \end{align*}
    \caption{Hyper-Coercion to Coercions}
    \label{fig:h2c}
  \end{subfigure}%
    \begin{subfigure}{.5\textwidth}
      \begin{align*}
        \interp{\Seq{\prj[\T_2]{\Lbl}}
                    {\Seq{\Int}{\inj[\T_3]}}}
               {\T_1} {\T_4}
        &=
               \hc{\prj{\Lbl}}{\T_2}
                  {\interp{\Int}{\T_2}{\T_3}}
                  {\T_3}{\inj}
      \\
      \interp{\Seq{\Int}{\inj[\T_2]}}{\T_1}{\T_3}
      &=
      \hc{\pmt}{\T_1}
         {\interp{\Int}{\T_1}{\T_2}}
         {\T_2}{\inj}
      \\
      \interp{\Seq{\prj[\T_1]{\Lbl}}{\Int}}
             {\T_1}{\T_3}
      &=
       \hc{\prj{\Lbl}}{\T_2}
          {\interp{\Int}{\T_2}{\T_3}}
         {\T_3}{\imt} 
      \\
      \interp{\Seq{\prj[\T_1]{\Lbl}}
                  {\failed{\T_2}{\Lbl_2}{\T_3}}}
             {\T_1}{\T_5}
      &=
      \hcfail{\prj{\Lbl_1}}{\T_2}{\T_3}{\Lbl_2}{\T_4}
      \\
      \interp{\failed{\T_2}{\Lbl_2}{\T_3}}
             {\T_1}{\T_4}
             &=
             \hcfail{\pmt}{\T_1}{\T_2}{\Lbl_2}{\T_3}
      \\
      \interp{\id[\T]}{\T}{\T} &= \id
      \\
      \interp{\fn{\HC_1}{\HC_2}}
             {\Fn{\T_1}{\T_2}}{\Fn{\T_3}{\T_4}}
             &=
      \fn{\interp{\HC_1}{\T_3}{\T_1}}
         {\interp{\HC_2}{\T_2}{\T_4}}
      \\
      \interp{\rf{\HC_1}{\HC_2}}
             {\Ref{\T_1}}{\Ref{\T_2}}
             &=
      \rf{\interp{\HC_1}{\T_2}{\T_1}}
         {\interp{\HC_2}{\T_1}{\T_2}}
    \end{align*}
    \caption{Coercion to Hyper-Coercion}
    \label{fig:c2h}
  \end{subfigure}
\end{figure}




\clearpage

\section{Schml's Representation of Hyper-Coercions}
\begin{figure}[tbh]
  \centering
  \begin{subfigure}{.5\textwidth}
    \[
    \begin{array}{llcl}
      \textbf{Intermediates}&
      \Med  & ::= & \id \mid \fn{\HC}{\HC} \mid \rf{\HC}{\HC} %
                  \mid \failed{\I}{\Lbl}{\I}\\
      \textbf{Hyper Coercions}&
      \HC   & ::= & \hc{\Prj}{\T}{\Med}{\T}{\Inj}
    \end{array}
    \]
    \caption{Syntax of Hyper Coercions (Alternate)}
    \label{fig:CompilerhcSyntax}
  \end{subfigure}%
  \begin{subfigure}{.5\textwidth}
    \begin{gather*}
      \inference{\T_1 \sim \I_1 & \I_2 \sim \T_2}
                {\cwt{\failed{\I_1}{\Lbl}{\I_2}}{\T_1}{\T_2}}
    \end{gather*}
    \caption{Typing Judgment (Alternate)}
    \label{fig:CompilerTyping}
  \end{subfigure}
  \begin{subfigure}{.35\textwidth}
    \begin{align*}
      \mkcrcn{\Dyn}{\Dyn} &= \; \hc{\prj{\Lbl}}{\Dyn}{\id}{\Dyn}{\imt}\\
      %
      \mkcrcn{\Dyn}{\I}   &= \; \hc{\prj{\Lbl}}{\I}{\id}{\I}{\imt} \\
      %
      \mkcrcn{\I}{\Dyn}   &= \hc{\pmt}{\I}{\id}{\I}{\inj} \\
      %
      \mkcrcn{\I_1}{\I_2} &= \hc{\pmt}{\I_1}{\mkIcrcn{\I_1}{\I_2}}{\I_2}{\imt}
    \end{align*}
    \caption{Make Coercion (Alternate)}
    \label{fig:makeCoercion}
  \end{subfigure}%
  \begin{subfigure}{.65\textwidth}
    \begin{align*}
      \mkIcrcn{\I}{\I}   &= \id\\
      %
      \mkIcrcn{\Fn[\!]{\T_{1}}{\T_{2}}}{\Fn[\!]{\T_{3}}{\T_{4}}} %
      &= \fn[\!]{\mkcrcn{\T_{3}}{\T_{1}}}{\mkcrcn{\T_{2}}{\T_{4}}}
      & \Fn[\!]{\T_{1}}{\T_{2}} \neq \Fn[\!]{\T_{3}}{\T_{4}} \\
      %
      \mkIcrcn{\Ref{\T_1}}{\Ref{\T_2}} %
      &= \rf{\mkcrcn{\T_{2}}{\T_{1}}}{\mkcrcn{\T_{1}}{\T_{2}}} %
      & \Ref{\T_1} \neq \Ref{\T_2} \\
      %
      \mkIcrcn{\I_1}{\I_2} &= \failed{\I_1}{\Lbl}{\I_2} %
      & \I_1 \sic \I_2
    \end{align*}
    \caption{Make Intermediate Coercion (Alternate)}
    \label{fig:makeIntCoercion}
  \end{subfigure}
  \begin{subfigure}{.5\textwidth}
    \begin{align*}
      %
      \cmp{\hc{\pmt}{\Dyn}{\id}{\Dyn}{\imt}}{\HC} &= \HC \\
      %
      \cmp{\hc{\Prj}{\I_1}{\Med}{\I_2}{\inj}} %
          {\hc{\pmt}{\Dyn}{\id}{\Dyn}{\imt}} %
      &=  \hc{\Prj}{\I_1}{\Med}{\I_2}{\inj} \\
      % Comp_hc_no_inj_prj
      \cmp{\hc{\Prj}{\I_1}{\Med_1}{\I_2}{\imt}} %
          {\hc{\pmt}{\I_2}{\Med_2}{\I_3}{\Inj}} %
      &=  \hc{\Prj}{\I_1}{\cmpm{\Med_1}{\Med_2}}{\I_3}{\Inj} \\
      % Comp_hc_inj_prj_ok
      \cmp{\hc{\Prj}{\I_1}{\Med_1}{\I_2}{\inj}} %
          {\hc{\prj{\Lbl}}{\I_3}{\Med_2}{\I_4}{\Inj}} %
      &=  \hc{\Prj}{\I_1}{\cmpm{\cmpm{\Med_1}{\mkIcrcn{\I_2}{\I_3}}}{\Med_2}}{\I_4}{\Inj} %
    \end{align*}
    \caption{Compose Hyper Coercions (Alternate)}
    \label{fig:composeHC}
  \end{subfigure}%
  \begin{subfigure}{.5\textwidth}
    \begin{align*}
      \cmpm{\id}{m} &= m\\
      \cmpm{m}{\id} &= m & m \neq \id \\
      \cmpm{\fn{\HC_1}{\HC_2}}{\fn{\HC_3}{\HC_4}} %
      &= \fn{\cmp{\HC_3}{\HC_1}}{\cmp{\HC_2}{\HC_4}}\\
      \cmpm{\rf{\HC_1}{\HC_2}}{\rf{\HC_3}{\HC_4}} %
      &= \rf{\cmp{\HC_3}{\HC_1}}{\cmp{\HC_2}{\HC_4}}\\
      \cmpm{\failed{\I}{\Lbl}{\I}}{m} &= \failed{\I}{\Lbl}{\I} %
      & m \neq \id \\
      \cmpm{m}{\failed{\I_1}{\Lbl_1}{\I_2}} &= \failed{\I}{\Lbl}{\I} %
      & m \neq \id \wedge m \neq \failed{\I_3}{\Lbl_2}{\I_4}
    \end{align*}
    \caption{Compose Intermediates (Alternate)}
    \label{fig:composeMed}
  \end{subfigure}


\end{figure}

\bibliographystyle{abbrvnat}
\bibliography{all}

\end{document}
